\documentclass[a4paper,titlepage,oneside]{article}

\usepackage{amsmath}
\usepackage{amsfonts}
\usepackage{amssymb}
\usepackage{amsthm}
\usepackage[utf8]{inputenc}
\usepackage[T1]{fontenc}
\usepackage{enumitem}
\usepackage{ngerman}
\usepackage{mathtools}
\usepackage{fontawesome}
%\usepackage{geometry}

%Formatting
\author{Tanja Kohler}
\title{Analysis für Informatik\small{ \\ - \\ Ass.Prof. Clemens Amstler}}
%\geometry{verbose,a4paper,tmargin=25mm,bmargin=25mm,lmargin=25mm,rmargin=25mm}


%Defining
\def\C{\ensuremath{\mathbb{C}} }
\def\N{\ensuremath{\mathbb{N}} }
\def\Q{\ensuremath{\mathbb{Q}} }
\def\Z{\ensuremath{\mathbb{Z}} }
\def\R{\ensuremath{\mathbb{R}} }
\def\im{\ensuremath{\imath} }
\def\e{\ensuremath{\mathit{e}}}
\def\WSP{\text{\faBolt}}

\newcommand{\alQ}[1]{\ensuremath{\quad#1&\quad}}
\newcommand{\fa}[1]{\ensuremath{\forall#1}}
\newcommand{\fain}[2]{\ensuremath{\forall#1\in#2}}
\newcommand{\ex}[1]{\ensuremath{\exists#1}}
\newcommand{\exin}[2]{\ensuremath{\exists#1\in#2}}
\newcommand{\abs}[1]{\ensuremath{\left|\:#1\:\right|}}

\def\sp{\hspace{0,1cm}}

\newcommand{\IA}[1]{\ensuremath{\text{\textbf{IA}: }#1\quad}}
\newcommand{\IV}{\ensuremath{\text{\textbf{IV}:}\quad}}
\newcommand{\IS}[1]{\ensuremath{\text{\textbf{IS}:}\quad #1\newline}}

\newcommand{\suminf}[2]{\ensuremath{\sum_{#1= 0}^{\infty}{\left(#2\right)}}}
\newcommand{\Suminf}[2]{\ensuremath{\sum_{#1=1}^{\infty}{\left(#2\right)}}}
%\newcommand{\liminf}[2]{\ensuremath{\lim\limits_{#1 \rightarrow \infty}{\left(#2\right)}}}
\newcommand{\Liminf}[2]{\ensuremath{\lim\limits_{#1 \rightarrow -\infty}{\left(#2\right)}}}
%\newcommand{\lim}[2]{\ensuremath{\lim\limits_{#1 \rightarrow 0}{\left(#2\right)}}}

\def\xor{\ensuremath{\text{ }\veebar}\text{ }}
\def\toinf{\ensuremath{\rightarrow \infty}}

\newtheoremstyle{thmstyle}{}{0.5cm}{}{}{\bfseries}{}{\newline}{\thmnumber{#2. }\thmname{#1}\quad\thmnote{ #3}\vspace{0.1cm}}

\theoremstyle{thmstyle}
\newtheorem{satz}{Satz}[subsection]
\newtheorem{korr}[satz]{Korollar}
\newtheorem{prop}[satz]{Proposition}
\newtheorem{defi}[satz]{Definition}
\newtheorem{bsp}[satz]{Beispiel}
\newtheorem{bem}[satz]{Bemerkung}

\renewcommand{\proofname}{\textbf{Beweis:}}
\renewcommand{\qedsymbol}{\textit{q.e.d.}}

%Documentstart
\begin{document}

\maketitle

\section{Reelle und Komplexe Zahlen}
\subsection{Reelle Zahlen}
Die reellen Zahlen \R erfüllen eine Reihe von Axiomen, die in drei Gruppen unterteilt werden können.

\begin{enumerate}[label=\Roman*.]
\item Algebraische Axiome
\item Anordnungsaxiome
\item Vollständigkeitsaxiome
\end{enumerate}

\subsubsection{Algebraische Axiome}
Die reellen Zahlen bilden mit der Addition \( + : \R \times \R \to \R \text{ mit } (a,b) \mapsto a + b\) und der Multiplikation \( * :  \R \times \R \to \R \text{ mit } (a,b) \mapsto a * b \)
einen Körper \((\R, +, * )\), der folgende Axiome erfüllt: 
\begin{enumerate}[label=\arabic*)]
\item $\R$ ist bzgl. der Addition eine Abelsche Gruppe. \((\R,+)\)
\item $\R \setminus \{0\}$ ist bzgl der Multiplikation eine Abelsche Gruppe. \((\R,*)\)
\item Das Distributivgesetz gilt: $ \fa a,b,c \in \R \sp a * (b + c) = a * b + a * c$
\end{enumerate}
Andere Beispiele von Körpern: \C, \Q, $\Z_p$ für $p$ prim.
Die Natürlichen Zahlen $ \N = \{1,\dots,\infty \} $ und die Ganzen Zahlen $\Z$ bilden keinen Körper.

\begin{prop}
$\fain{x}{\R} \text{ gilt } 0 * a = 0$.
\begin{proof}
\begin{align*}
0+0 \alQ{=} 0 \Rightarrow \\
a ( 0 + 0 )\alQ{=} a * 0 \overset{\text{Distributivgesetz}}{\Rightarrow} \\
a * 0 + a * 0 \alQ{=} a * 0  \overset{\text{assiozativ}}{\Rightarrow} \\
a * 0 + (a * 0 - a * 0) \alQ{=} (a * 0 - a * 0) \overset{\text{additives Inverses}}{\Rightarrow} \\
a * 0 +  0 \overset{0+0=0}{=} a * 0 \alQ{=} 0.
\end{align*}
\end{proof}
\end{prop}

\begin{defi}[Potenzschreibweise]
Für $a \in \R $ und $n \in \Z$ wird $a^n$ folgendermapen induktiv definiert:
\begin{itemize}
\item $a^0 = 1$ 
\item $\fa n > 1 \quad a^{n+1} = a * a^n $
\item $ \fa n > 1 \sp \fa a \ne 0 \quad a^{-n} = \left(a^{-1}\right)^n$
\end{itemize}
\end{defi}
\newpage
\begin{bem}
\fain{ a, b}{\R \setminus \{0\}} und \fain{n, m}{\Z} gilt:
\begin{enumerate}[label=(\arabic*)]
\item \(a^n * a^m = a^{n+m} \)
\item \(a^{n^m} = a^{n * m} \)
\item \(a^n * b^n = (a * b)^n \)
\end{enumerate}
\begin{proof}
\begin{math}\\
\text{(1) } a^n * a^m \overset{\text{n. Def.}}{=} \overbrace{a \dots a}^{n\text{-mal}}*\overbrace{a \dots a}^{m\text{-mal}} = \overbrace{a \dots a}^{n+m\text{-mal}} \overset{\text{n. Def.}}{=} a^{n+m}\\
\text{(2) }  a^{n^m} = a^{\overbrace{n \dots n}^{m\text{-mal}}} = a^{m * n} = a^{n*m}\\
\text{(3) }  a^n * b^n = \overbrace{a \dots a}^{n\text{-mal}}*\overbrace{b \dots b}^{n\text{-mal}} = \overbrace{a \dots a b \dots b}^{n\text{-mal}} = (a*b)^{n}\\
\end{math}
\end{proof}
\end{bem}

\subsubsection{Anordnungsaxiome}
Die reellen Zahlen werden in positive Zahlen ($x > 0$), negative Zahlen ($x < 0$) und 0 ($x = 0$) unterteilt. Dabei ist $x < 0 \Leftrightarrow -x > 0$
Und es gelten folgende Axiome:
\begin{enumerate}[label=(\arabic*)]
\item \(\fa x \in \R \) gilt genau eine der folgenden Bedingungen: \(x > 0\text{, } x = 0\text{, } x < 0 \)
\item \(\fa x,b \in \R \sp x,b > 0 \text{ gilt: } a + b > 0 \land a * b > 0\)
\end{enumerate}
Wir schreiben für $a, b \in \R \sp a > b \Leftrightarrow a - b > 0\text{ und }a \ge b \Leftrightarrow a > b \lor a = b $

\begin{prop}
\fain{a, b}{\R} gilt: $a < b$ und $b < c \Rightarrow a < c$
\begin{proof}
selbst  % TODO
\end{proof}
\end{prop}

\begin{bem}
\fain{a, b, c}{\R} gilt:
\begin{enumerate}[label=\alph*)]
\item $a < b \Rightarrow a + c < b + c$
\item $a < b$ und $c > 0 \Rightarrow a * c < b * c$
\item $a < b$ und $c < 0 \Rightarrow a * c > b * c$
\item $a \ne 0 \Rightarrow a^2 > 0$  speziell $1 > 0$
\item $0 < a < b$ und $a < b < 1 \Rightarrow b^{-1} < a^{-1}$
\end{enumerate}
\end{bem}

\begin{defi}
Für $a \in \R$ und der Betrag \abs{a} folgendermaßen definiert. 
\[\abs{a} = \begin{cases}
 a & \text{wenn } a > 0\\
-a & \text{wenn } a < 0
\end{cases}\]
\end{defi}

\begin{satz}
\fain{ b}{ \R} gilt:
\begin{enumerate}[label=(\arabic*)]
\item $\abs{a * b} = \abs{a} * \abs{b}$
\item $\abs{a + b} \le \abs{a} + \abs{b} $ (Dreiecksungleichung)
\item $\abs{a - b} \ge \abs{\abs{a} - \abs{b}}$ (umgekehrte Dreiecksungleichung)
\end{enumerate}
\begin{proof}.
\begin{enumerate}[label=(\arabic*)]
\item siehe PS
\item 
\begin{itemize}
\item $a \le \abs{a} \text{ und } b \le \abs{b} \Rightarrow a + b \le \abs{b} + b \le \abs{a}+\abs{b} $
\item $-a \le \abs{a} und -b \le \abs{b}  \Rightarrow -(a + b) = -a + -b \le \abs{a} + -b \le \abs{a} +\abs{b}  $
\end{itemize}
$\Rightarrow \text{ d.h. } a + b \le \abs{a} + \abs{b}  \text{ und } -(a + b) \le \abs{a} + \abs{b}  \Rightarrow \abs{a + b} \le \abs{a} + \abs{b} $
\item siehe PS
\end{enumerate}
\end{proof}
\end{satz}

\begin{bem}[Archimedisches Axiom]
Für zwei positive Zahlen, $a, b$ gibt es immer eine natürliche Zahl $n$, sodass folgendes gilt: \(n * b > a\) Also:
\[\fa a,b > 0 \sp \ex n \in \N \quad n * b > a\]
Als Folgerung erhalten wir: Setze $b = 1$
\[\fa a > 0 \sp \ex n \in \N \quad n > a\]
\end{bem}

\begin{satz}[Bernoullische Ungleichung]
Sei $a > -1$ dann gilt $\fain{n}{\N} : (1+a)^n >= 1 + na$
\begin{proof}
Übung.  %TODO
\end{proof}
\end{satz}

\begin{korr}
Sei $a > 0$. 
\begin{enumerate}[label=(\arabic*)]
\item Ist \(a > 1 \sp \forall k > 0 \sp \exists n \in \N \text{, sodass }a^n > k.\)
\item \(0 < a < 1 \sp \forall \epsilon > 0 \sp \exists n \in \N \text{, sodass }a^n < \epsilon\)
\end{enumerate}
\begin{proof}
\sp
\begin{enumerate}[label=(\arabic*)]
\item \( a > 1 \Rightarrow \exists x \in \R x = a - 1 > 0 \Rightarrow a = x +1 \Rightarrow a^n = (x + 1)^n \Rightarrow \\
(x + 1)^n \underset{\text{Bernoulli}}{\ge}1 + nx  > 1 + k -1 = k \text{, da }\forall n \in \N \sp \exists x > 0 \text{ mit } nx > k - 1\)
\item Sei \( 0 < a < 1 \text{, sei } b = \frac{1}{a} > 1\text{,} \underset{\text{mit }(1)}{\Rightarrow} \exists k \in \R \text{ mit } b^n > k = \frac{1}{\epsilon} \Rightarrow \\
\left(\frac{1}{a}\right)^n > \frac{1}{\epsilon} \Rightarrow  \frac{1}{a^n} > \frac{1}{\epsilon} \Rightarrow a^n < \epsilon.\)
\end{enumerate}
\end{proof}
\end{korr}

\subsubsection{Vollständigkeitsaxiom}
Die Zahlengerade \R hat keine Lücken. 

\begin{defi}
Sei $ M \subset \R$ eine Teilmenge.\begin{enumerate}
\item $k \in \R $ heißt obere Schranke von $M$ wenn gilt, $\fa x \in M, x \le k$. $M$ heißt nach oben beschränkt, wenn es eine obere Schranke gibt.
zB $\N$ ist nicht nach oben beskchränkt, nach dem Archimedischem Axiom.
\item $k \in \R $ heißt untere Schranke von $M$ wenn gilt, $\fa x \in M, x \ge k$. $M$ heißt nach unten beschränkt, wenn es eine untere Schranke gibt.
\item $M$ heißt beschränkt, wenn eine obere und untere Schranke existiert. 
(äquivalente Definition für Beschränktheit: $\exists k \in \R , |x| \le k \fa x \in M$)
\item $a \in \R$ heißt Infimum von $M$, falls $a$ größte untere Schranke von $M$ ist. d.h. $a$ ist untere Schranke von $M$ und ist $k$ eine untere schranke von $M$, dann folgt $k \le a$
\[\text{Schreibweise: } a = inf(M)\]
\item $b \in \R$ heißt Supremum von $M$, falls $b$ kleinste obere Schranke von $M$ ist. d.h. $b$ ist obere schranke von $M$ und ist $k$ eine obere schranke von $M$, dann folgt $k \ge a$
\[\text{Schreibweise: } b = sup(M)\]
\end{enumerate}
\end{defi}

\begin{bsp}
Sei a < b dann ist $inf( [a,b] ) = a = inf( (a,b) )$ und $ sup( [a,b] ) = b = sub( (a,b) )$.
\[[a, b] = \{a \in \R | a \le x \le b\} \text{ heißt abgeschlossenes Intervall}\]
\[(a, b) = \{a \in \R | a < x < b\} \text{ heißt offenes Intervall}\]
\end{bsp}

\begin{bem}[zur Erinnerung]
Definition der natürlichen Zahlen (Axiom des kleinsten Element (Pianoaxiome)) \\
Jede Teilmenge der natürlichen Zahlen hat ein kleinstes Element.
\end{bem}

\begin{satz}[Vollständigkeitsaxiom]
Jede nicht leere, nach unten beschränkte Teilmenge $M \subset \R$  besitzt ein Infimum $inf(M) \in \R$.
\begin{proof}[ohne Beweis] \end{proof}
\end{satz}

\begin{bem}
$inf(M) \text{ muss kein Element von } M$ sein.
\end{bem}

\begin{prop}
Jede nicht leere nach oben bescrhänkte Teilmenge $M \subset \R$ besitzt ein Supremum $sup(M) \in \R$.
\begin{proof}
Sei $M$ nach oben beschränkt. Sei $-M = \{-< | x \in M\}$. Sei $a$ eine obere Schranke von $M$. d.h. $\fa x \in M  x \le a \Rightarrow -a \le -x \fa x \in M \Rightarrow $ d.h. $-a$ ist untere Schranke von $-M$. d.h. $-M$ ist nach unten beschränkt. Nach dem Vollständigkeitsaxiom, existiert ein Infimum. Sei $b= inf(-M) \Rightarrow -a \le b \Rightarrow -b \le a und b \le -x \Rightarrow x \le -b$.Also $-b$ ist obere Schranke und kleinste obere Schanke. $ \Rightarrow -b = sup(M)$ 
\end{proof}
\end{prop}

\begin{prop}
$sup(M)$ und $inf(M)$ sind eindeutig bestimmt.
\begin{proof}
Seien $m$ und $m'$ Suprema von $M \Rightarrow m \le m'$ und $m' \le m \Rightarrow m = m'$.\\
analog für Infimum.
\end{proof}
\end{prop}

\subsection{Komplexe Zahlen}
Die Menge der komplexen Zahlen \C sind die Punkte der Ebene \(\R^2 = \{(a,b) : a,b \in \R\}\)
\[(a,b) = (a, 0) + (0, b) = a (1,0) + b (0, 1)\]
Wir setzen \(1 = (1,0), \im = (0,1) \Rightarrow (a, b) = a + \im b\)

[grafik R2 mit z = (a, b) und (1,0) und (0,1) eingezeichnet]  % TODO

zusätzlkich verlangen wir \(\im^2 = -1\) Also: \[\C := \{z = a + \im b | a, b \in \R, \im^2 = -1 \}\]

\begin{satz}
Es gilt:  \C ist ein Körper.
\begin{proof} nachrechnen. \end{proof} % TODO
\end{satz}

\begin{defi}
Sei \( z = a + \im b \in \C \), dann heißt \(\bar{z} = a - \im b \) die konjungiert komplexe Zahl von $z$. \\
\(\abs{z} = \sqrt{z * \bar{z}} = \sqrt{a^2 + b^2}\) heißt Betrag von \(z\)
$a = Re(z)$ heißt Realteil von $z$
$b = Im(z)$ heißt Imaginärteil von $z$
\end{defi}

\begin{satz}
Es gilt: $Re(z) = \frac{z + \bar{z}}{2}$ und $Im(z) = \frac{z - \bar{z}}{2\im}$.
\begin{proof} selber mit grafik. % TODO
\end{proof}
\end{satz}

\begin{prop}
Es gilt:
\begin{enumerate}[label=(\roman*)]
\item \( \bar{\bar{z}} = z, \bar{z_1} + \bar{z_2} = \bar{z_1 + z_2}, \bar{z_1} * \bar{z_2} = \bar{z_1 * z_2}, \abs{\bar{z}} = \abs{z} \)
\item \( \abs{z} \ge 0, \abs{z} = 0 \Leftrightarrow z = 0\)
\item \(\abs{z_1 z_2} = \abs{z_1} \abs{z_2}\)
\item \(\abs{z_1 + z_2} \le \abs{z_1} + \abs{z_2}\)
\end{enumerate}
\begin{proof}
\begin{enumerate}[label=(\roman*)]
\item nachrechnen. % TODO
\item nachrechnen. % TODO
\item \(\abs{z_1 z_2}^2 = (z_1 z_2)(\bar{z_1 z_2}) =  .... \) \\   % TODO
\item Sei $a, b \in \R \Rightarrow a^2 \le a^2 + b^2 \Rightarrow a \le \sqrt{a^2 + b^2}$ \\
Sei $ z = a + \im b , a = Re(z) \Rightarrow \abs{Re(z)} \le z \Rightarrow Re(z_1 \bar{z_2}) \le \abs{Re(z_1 \bar{z_2}} \le \abs{z_1 \bar{z_2}} = \abs{z_1}\abs{\bar{z_2}} = \abs{z_1}\abs{z_2}$
$\Rightarrow \abs{z_1 + z_2}^2 = (z_1 + z_2)\bar{(z_1 + z_2)} = (z_1 + z_2)(\bar{z_1} + \bar{z_2}) = z_1\bar{z_1} + z_2\bar{z_1} + z_1\bar{z_2} z_2\bar{z_2} = \abs{z_1}^2 + z_1\bar{z_2} + \bar{z_1\bar{z_2}} + \abs{z_2}^2 = |z_1|^2 + 2 Re (z1 \bar{z2}) + |z_2|^2 \le $ realteil ersetzen nach oben und dann hamma quadrate , mit wurzel ziehen folgt die aussage. % TODO
\end{enumerate}
\end{proof}
\end{prop}

% AB HIER FORMATIEREN!!! und Korrigieren
\section{Folgen und Reihen}
\subsection{Folgen}

\begin{bsp}
Betrachte [Rechtwinkliges Dreieck mit beschriftung 1, 1, und $\sqrt{2}$]
\( \Rightarrow \sqrt{2} \in \R\), aber \(\sqrt{2} \not\in \Q\)
\end{bsp}
\begin{proof}
indirekter Beweis: Angenommen \(\sqrt{2} \in \Q\), d.h. 
$\sqrt{2} = \frac{p}{q}\text{ mit }p \in \Z, q \in \N.$
Wir können annehmen, dass p und q nicht beide durch $2$ teilbar sind (sonst kürzen wir.) teilerfremd.
\(\Rightarrow 2 = \frac{p^2}{q^2} \Rightarrow p^2 = 2q^2 \Rightarrow 2 | p^2 \Rightarrow 2 | p \Rightarrow \exists m \text{ mit } p = 2 m.\)
\( \Rightarrow 2q^2 = (2m)^2 = 4m^2 \Rightarrow q^2 = 2m^2 \text{ d.h. } 2 | q^2  \Rightarrow 2 | q \) Also p und q sind beide durch $2$ teilbar.  \WSP p und q sind teilerfremd. $\Rightarrow \sqrt{2} \not\in \Q$
\end{proof}

\begin{bem}
\( \sqrt{2} \) ist die positive Lösung von \(a^2 = 2 \Leftrightarrow a = \frac{2}{a} \Leftrightarrow  2a = a + \frac{2}{a} \Leftrightarrow a= \frac{1}{2}\left(a + \frac{2}{a}\right)\)
Betrachte die rechte Seite dieser Gleichung und berechne diese induktiv
Setze zB \[a_1 = 1 \] \[a_{n+1} = \frac{1}{2}\left(a_n + \frac{2}{a_n}\right)\]
\(a_1 = 1
a_2 = ... = 1,5
a_3 = ... \approx 1.41
a_3 = ... \approx 1,4142\)
...

Also $a_n$ nähert sich mit wachsendem n immer mehr an $\sqrt{2}$
Dies führt zu dem Begriff \textbf{Grenzwert einer Folge}.
\end{bem}

\begin{defi}
Eine Folge $(a_n)_{k=0}^{\infty}$ reeller Zahlen ist eine Abbildung $\N_0 \rightarrow \R\text{ mit } n \mapsto a_n$ 
Bezeichnung: Wir schreiben für Folgen
\begin{itemize}
\item $(a_n)_{k=0}^{\infty}$
\item $(a_n)_{n\ge0}$
\item $(a_n)_{n\in\N}$
\item $(a_n)$
\item $(a_n)_{n\ge n_0}$ für $n_0 \in \N$
\end{itemize}
\end{defi}

\begin{defi}
\begin{enumerate}
\item Eine Folge $(a_n)$ heißt (streng) monoton wachsend, wenn \fain{a}{\N_0} $a_n \le a_{n+1} (a_n < a_{n+1})$ gilt. Schreibweise: $(a_n)\nearrow$, $(a_n)\uparrow$.
\item Eine Folge $(a_n)$ heißt (streng) monoton fallend, wenn \fain{a}{\N} $a_n \ge a_{n+1} (a_n > a_{n+1})$ gilt. Schreibweise: $(a_n)\searrow$, $(a_n)\downarrow$.
\item Eine Folge $(a_n)$ heißt (streng) monoton, sie (streng) monoton wachsend oder fallend ist.
\end{enumerate}
\end{defi}

\begin{bsp}
\begin{enumerate}
\item Die konstante Folge $a_n := a$ ist monoton fallend und steigend.
\item Die harmonische Folge $a_n := \frac{1}{n} \fa{n\ge1}$ ist streng monoton fallend.
\item Die alternierende Folge $a_n := (-1)^n$ ist nicht monoton.
\item Die geometische Folge, Sei $a\in\R a_n := a^n \fa{n\ge0}$ ist \(\begin{cases}
\text{streng monoton wachsend} & \text{wenn } a > 0\\
\text{streng monoton fallend} & \text{wenn } 0 < a < 1\\
\text{monoton fallend und steigend} & \text{wenn } a = 1\\
\text{nicht monoton} & \text{sonst}
\end{cases} \)
\item Die Fibonacci Folge \(f_0 = 0, f_1 = 1, f_n = f_{n-1} + f_{n-2} \fa{n \ge 2}\) ist monoton wachsend.
\end{enumerate}
\end{bsp}

\begin{defi}[der Konvergenz]
Eine Folge zeeller Zahlen $(a_n)_{n\in\N}$ heißt konvergent gegen $ a\in\R$ (Schreibweise. ${lim}_{n\rightarrow\infty}a_n = a, lim a_n = a$
wenn es für jedes $\epsilon > 0$ ein $N = N(\epsilon) \in \N$ gibt, sodass für alle $n \ge N$ die Ungleichung $\abs{a_n - a} < \epsilon$ gilt.
$a$ heißt der Grenzwert (Limes) der Folge $(a_n)$
Die Folge $(a_n)$ heißt divergent, wenn sie nicht konvergent ist.
\[\text{Also: }lim a_n = a \Leftrightarrow \fa{\epsilon > 0} \exin{N(\epsilon)}{\N} : \fa{n\ge N} \abs{a_n - a} < \epsilon\]
\end{defi}

\begin{bem}
\begin{enumerate}
\item Sei $a \in \R, \epsilon > 0$ $U_\epsilon(a) := (a-\epsilon, a+\epsilon) = \{x \in \R | a - \epsilon < x < a + \epsilon\}$ heißt $\epsilon$-Umgebung von $a$. \[ a_n \in U_\epsilon(a) \Leftrightarrow a-\epsilon < a_n < a + \epsilon \Leftrightarrow -\epsilon < a_n - a < \epsilon \Leftrightarrow \abs{a_n - a} < \epsilon\]
Also Die Folge $(a_n)$ konvergiert gegen $a \Leftrightarrow $ Die Folgenglieder $a_n$ liegen ab einer Schwelle $N$ alle in der $\epsilon$-Umgebung von $a$.
\item $(a_n)$ konvergiert nicht gegen $a \Leftrightarrow \exists \epsilon > 0 \forall N\in\N \exists n \ge N |a_n - a| \ge \epsilon$.
\end{enumerate}
\end{bem}

\begin{bsp}
1) Die harmonische Folge $a_n = \frac{1}{n} \Rightarrow \lim\limits_{n \rightarrow \infty}{\frac{1}{n}} = 0$
\begin{proof}
Sei $\epsilon > 0$ Wähle $ N > \frac{1}{\epsilon}$
$\abs{a_n - 0}  = \abs{\frac{1}{n} - 0} = \frac{1}{n} <= \frac{1}{N} < \epsilon$
\end{proof}
2) Die alternierende Folge $a_n = (-1)^n$
\begin{proof}
Angenommen $\exists a in \R$ mit $ \lim{a_n} = a$ Wähle $\epsilon > 0 \Rightarrow \exists N \in  \N \forall n \ge N \abs{b_n - a} < \frac{1}{2}$. Da $b_{n+1} - b_n = +- 2$ ist $ \forall n \ge N 2 = \abs{b_{n+1} - b_{n}} = \abs{b_{n+1} - a - (b_n - a)} \le \abs{b_{b+1} - a} + \abs{b_n - a} < \frac{1}{2} + \frac{1}{2} = 1 \Rightarrow 2 < 1 \WSP \Rightarrow (b_n) $ ist divergent.
\end{proof}
3) Die geometsiche Folge $(a^n)_{n\ge1}$
1. Fall $|a| < 1 \Rightarrow \lim\limits_{n \toinf}a^n = 0$
\begin{proof}
Sei $\epsilon < 0 \overset{archim}{\Rightarrow} \exists N\in \N \abs{a}^N < \epsilon \Rightarrow \forall n \ge N : |a^n - 0| = |a|^n \le |a|^N < \epsilon$
\end{proof}
2. Fall $a = 1 \Rightarrow a^n = 1 \Rightarrow \lim{a^n} = 1$
3. Fall $a = -1 \Rightarrow $ divergent weil alternierend.
4. Fall $|a| > 1 \forall K > 0 \exists n \in \N |a|^n > K$ d.h. $(a^n)$ ist unbeschränkt.
\end{bsp}

\begin{defi}
Eine Folge $(a_n)$ heißt nach oben (unten) beschränkt, wenn es ein $A \in \R$ gibt mit $\forall n \in \N a_n \le A (a_n \ge A)$.
$(a_n)$ heißt beschränkt, wenn $(a_n)$  nach oben oder unten beschränkt ist. d.h. $\exists K \in \R |a_n| \le K \forall n \in \N$
\end{defi}

\begin{satz}
Jede konvergente Folge $(a_n)$ ist beschränkt.
\end{satz}
\begin{proof}
\begin{math}
\lim\limits_{n \toinf}a_n = a\text{. Wähle }\epsilon = 1 > 0 \Rightarrow \exists N \in \N : \forall n \ge N \abs{a_n - a} < 1 \Rightarrow |a_n| = |a + (a_n - a) | \le |a| + |a_n- <| = |a| + 1 \forall n \ge N\text{. Sei } K = max\{|a_1|,  |a_2|, \dots, |a_n-1|, |a|+1\} \Rightarrow |a_n| < k \forall n \ge 1 
\end{math}
\end{proof}

\begin{bem}
Die uMkehrung gilt nicht. d.h. eine beschränkte Folge ist nicht konvergent. siehe die alternierende Folge.
\end{bem}


\begin{satz}[Monotoniekriterium]
(1) Jede monoton wachsende nach oben beschränkte Folge ist konvergent.\\
(2) Jede monoton fallende nach unten beschränkte Folge ist konvergent.\\
Bem: Das Monotonie-Kriterium ist äquivalent zur Vollständigkeit.
\end{satz}
\begin{proof}
(1) Sei $(a_n)$ monoton wachsen und nach oben beschränkt. Nach dem Vollständigkeitsaxiom existiert $a := sup\{a_n | n \in \N\}$.
Sei $\epsilon > 0 \Rightarrow a - \epsilon$ ist keine obere Schranke von $\{a_n | n \in \N\} \Rightarrow \exists N \in \N : a-\epsilon < a_N \le a$. Da $(a_n)$ monoton wachsend ist, $ \Rightarrow \forall n \ge N a_N \le a_n \Rightarrow a-\epsilon < a_N \le a_n \le a < a+\epsilon \forall n \ge N \Rightarrow
a-\epsilon < a_n < a+\epsilon \forall n \ge N \Rightarrow |a_n - a| < \epsilon \forall n \ge N \Rightarrow \lim{a_k} = a.$
(2) analog.
\end{proof}

\begin{satz}
Der Grenzwert einer Folge ist eindeutig bestimmt.
\end{satz}
\begin{proof}
Angenommen $\lim\limits_{n \toinf}{a_n} = a$ und $\lim\limits_{n \toinf}{a_n} = b$ und $ a \ne b$.\\
Sei $\epsilon = \frac{1}{2}\left|b-a\right| \Rightarrow \exists N_1 \forall n \ge N_1 : |a_n - a| < \epsilon$ und $\Rightarrow \exists N_2 \forall n \ge N_2 : |a_n - b| < \epsilon$. Sei $N := max\{N_1, N_2\}$.\\
$\Rightarrow n \ge N : |b-a| = |(b-a_n) + (a_n - a)| \le |b - a_n| + |a_n - a| = |a_n - b| + |a_n - a| < \frac{1}{2}\left|b-a\right| + \frac{1}{2}\left|b-a\right| = |b-a|  \Rightarrow |b-a|  < |b-a| \WSP \Rightarrow a = b$
\end{proof}


%% KORRIGIERE AB HIER!!
\begin{satz}[Rechenregeln für konvergente Folgen]
Seien $(a_n)$ und $(b_n)$ zwei konvergente Folgen. Dann gilt:
\begin{enumerate}
\item $(a_n \pm b_n)$ ist konvergent und $\lim\limits_{n \toinf}{a_n  \pm  b_n} = \lim\limits_{n \toinf}{a_n}  \pm \lim\limits_{n \toinf}{b_n}$.
\item $\lambda (a_n)$ ist konvergent und $\lim\limits_{n \toinf}{\lambda a_n} = \lambda \lim\limits_{n \toinf}{a_n}$.
\item $(a_n b_n)$ ist konvergent und $\lim\limits_{n \toinf}{a_n b_n} = \lim\limits_{n \toinf}{a_n} \lim\limits_{n \toinf}{b_n}$
\item Ist $(b_n) \ne 0 \forall n \ge n_0 und \lim\limits_{n \toinf}{b_n} \ne 0$ Dann $(\frac{a_n}{b_n})$ ist konvergent und $\lim\limits_{n \toinf}{\frac{a_n}{b_n}} = \frac{\lim\limits_{n \toinf}{a_n}}{\lim\limits_{n \toinf}{b_n}}$.
\item $a_n \le b_n$ dann ist $\lim{a} \le \lim{b} \forall n \ge n_0$.
\end{enumerate}
\begin{proof}
Sei $\lim{a_n} = 0\text{ und } \lim{b_n} = b$.\\
1) Sei $\epsilon > 0 \Rightarrow \exists N_1, N_2, \in \N \\
\abs{a_n - a} < \frac{\epsilon}{2} \forall n \ge N_1\\
\abs{b_n - b} < \frac{\epsilon}{2} \forall n \ge N_2\\
\Rightarrow \forall n \ge max{N_1, N_2}\\
\abs{(a_n + b_n) - (a + b)} =\abs{(a_n - a) + (b_n + b)} \le \abs{(a_n - a)}+ \abs{(b_n + b)} < \frac{\epsilon}{2} + \frac{\epsilon}{2} = \epsilon.
\Rightarrow beschränkt und \lim{a_n+b_n} = a+b$.
analog für $-$
\\\\2) Übung
\\\\3)
Jede konvergente Folge ist beschränkt $\Rightarrow \exists K \in \R$ mit $|a_K| \le K$ und $|b| \le K$\\
Sei $\epsilon > 0 \Rightarrow \exists N_1, N_2 \in \N |a_n - a| < \frac{\epsilon}{2K}$ und $|b_n - b| < \frac{\epsilon}{2K}$.
$\Rightarrow \forall n \ge max{N_1, N_2} gilt |a_n b_n - a b | = |a_n b_n - a_n b + a_n b + a b| = | a_n (b_n - b) + b (a_n - a)| \le | a_n (b_n - b)| + |b (a_n - a)| = \underbrace{| a_n|}_{\le K} (b_n - b)| + \underbrace{|b|}_{\le K}|(a_n - a)| < K \frac{\epsilon}{2K} + K \frac{\epsilon}{2K} = \epsilon
\\\\4)
Zeige \lim\limits_{n \toinf}{\frac{1}{b_n}} = \frac{1}{\lim\limits_{n \toinf}{b_n}}
|| b_n | - |b| | \le | b_n - b| < \frac{|b|}{2} \forall n \ge n_0
\Rightarrow -\frac{|b|}{2} < |b_n| - |b| < \frac{|b|}{2}
\Rightarrow \frac{|b|}{2} < |b_n| \Rightarrow \frac{1}{|b_n|} < \frac{2}{|b|} für alle n größer n_0
Sei epsilon > 0 folgt esgibt N für alle n größergleich N
|b_n - b| < (\epsilon |b| ^2)/ 2
\Rightarrow \abs{\frac{1}{b_n} - \frac{1}{b}} = aufgleichen nenner = \frac{1}{|b_n|}\frac{1}{|b|}  | rest in klammer |  mit epsilon nach voraussetzungen abschätzen
\\\\ 5) 
Sei a_n \le b_n \forall n \ge n_0. zz. a \le b
Angenommen a > b
Sei \epsilon \frac{a-b}{2} > 0 \Rightarrow \exists N_1, N_2 \in \N
\abs{a_n - a} < \epsilon  \forall n \ge N_1
\abs{b_n - b} < \epsilon  \forall n \ge N_2 
\Rightarrow \forall n \ge max{N_1, N_2}
b_n < b + \epsilon = b + (a-b) / 2 = (2b + a - b)/2 = (b + a ) /2 = (2a - a + b) / 2 = a - (a-b)/ 2 = a - \epsilon < a_n 
\Rightarrow  b_n < a_n \forall n \ge max{N_1, N_2}  \WSP 
\Rightarrow a \le b$
\end{proof}
\end{satz}


\begin{satz}[Sandwich-Theorem]
Sei $(a_n)$ und $(b_n)$ zwei konvergente Folgen mit der Eigenschaft, dass $\lim{a_n} = \lim{b_n} = a$.
Sei $(c_n)$ eine Folge mit der Eigenschaft, dass $a_n \le c_n \le b_n \forall n \ge n_0$
Dann ist $(c_n)$ konvergent und $\lim{c_n} = a$.
\begin{proof}\begin{math}
Sei \epsilon > 0 \Rightarrow \exists  N_1, N_2 \in \N
a-\epsilon < a_n < a + \epsilon  für alle n >= N1
a-\epsilon < b_n < a + \epsilon  für alle n >= N2
\Rightarrow für alle n aus max {n1, N2} 
gilt a-\epsilon < a_n <= c_n <= b_n < a + \epsilon  für alle n >= N
\Rightarrow |c_n - a| < \epsilon \Rightarrow \lim{c_n} = a
\end{math}\end{proof}
\end{satz}

\begin{bsp}
\begin{enumerate}
\item Sei $(a_n)$ eine Folge mit $0 \le a_n \le \frac{1}{n} \Rightarrow \lim{a_n} = 0$
\item $a_n = \sqrt{2n} - \sqrt{n}$ ist divergent, denn $a_n = \left(\sqrt{2n} - \sqrt{n}\right)\frac{\sqrt{2n} + \sqrt{n}}{\sqrt{2n} + \sqrt{n}} 
= \frac{2_n - n}{\sqrt{n}*\left(\sqrt{2} - 1\right)} \underbrace{\ge}_{\left(\sqrt{2} - 1\right)\le 3} \frac{n}{3\sqrt{n}} = \frac{sqrt{n}}{3} \rightarrow^{n-> \infty} \infty.$
\end{enumerate}
\end{bsp}

\begin{defi}
Eine Folge $(a_n)$ heißt bestimmt divergent gegen $+ \infty $ ($- \infty$) wenn gilt:
$\forall K \in \R \exists N \in \N \quad \forall n \ge N a_n > K (bzw a_n < K)$\\
Wir schreiben: $\lim{a_n} = +\infty \text{ (bzw }-\infty)$\\
nach oben unbeschränkt oder nach unten unbeschränkt
\end{defi}

\begin{bsp}
\begin{enumerate}
\item Die Fibonacci Folge ist bestimmt divergent gegen $+ \infty$
\item Sei $a_n = n$, dann folgt $\lim{a_n} = \infty$
\item Sei $\lim{a_n} = \infty \Leftrightarrow \lim{-a_n} = - \infty$
\item Die Folge $a_n = (-1)^n$ ist divergent aber nicht bestimmt divergent.
\item Sei $(a_n)$ bestimmt divergent und $a_n \ne 0 \forall n \ge n_0$, dann folgt $\lim{\frac{1}{a_n}} = 0.$
\end{enumerate}
\begin{proof}
Sei $\lim{a_n} = \infty \Rightarrow \forall \epsilon > 0 \exists N \in \N \forall n \ge N
a_n > \frac{1}{\epsilon} > 0 \Rightarrow \frac{1}{a_n} < \epsilon \Rightarrow \abs{\frac{1}{a_n} - 0} < \epsilon$, da $a_n > 0 \Rightarrow \lim{\frac{1}{a_n}} = 0$
\end{proof}
\end{bsp}


\begin{defi}
Sei $(a_n)$ eine Folge reeller Zahlen, und $n_0 < n_1 < n_2 <... < n_k < .....$ eine Teilmenge der Natürlichen Zahlen.
Dann heißt die Folge $(a_{n_k})_{k\in\N}$ eine Teilfolge von $(a_n)_{n\in\N}$
\end{defi}

\begin{bem}
Ist die Folge $(a_n)$ konvergent, dann ist auch jede Teilfolge von $(a_n)$ konvergent.
\begin{proof} Übung
\end{proof}
\end{bem}

\begin{defi}
Sei $(a_n)$ eine Folge. Eine Zahl $a \in \R$ heißt Häufungspunkt (Häufungswert) der Folge $(a_n)$, wenn es eine Teilfolge von $(a_n)$ gibt die gegen $a$ konvergiert.
\end{defi}

\begin{bem}
\begin{enumerate}
\item Sei $\lim{a_n} = a$, dann ist $a$ der einzige Häufungspunkt der Folge $(a_n)$.
\item Eine bestimmt divergente Folge hat keinen Häufungspunkt.
\item Die Folge $a_n = (1/n)+ (-1)^n$ besitzt die zwei Häufungspunkte $-1$ und $+1$. $\lim{a_{2n}} = \lim{\frac{1}{2n} (-1)^{2n}} = \lim{\frac{1}{2n}} + 1 = 1 \text{ und } \lim{a_{2n+1}} = \lim{\frac{1}{2n+1} (-1)^{2n+1}} = \lim{\frac{1}{2n+1}} - 1 = - 1$
\item Jede konvergente Folge ist beschränkt, aber jede beschränkte Folge muss nicht konvergent sein.
\end{enumerate}
\end{bem}

Folgender Satz ist äquvivalent zum Vollstaändigkeitsaxiom:
\begin{satz}[von Bolzano-Weierstraß]
Jede beschränkte Folge reeller Zahlen besitzt eine konvergente Teilfolge.
\begin{proof}
\((a_n)_{n \in \N_0} \) ist beschränkt, d.h. $\exists A \in \R$ mit $-A \le a_n \le A \forall n \ge 0$\\
Sei $A_k = \{ a_m : m \ge k\} \Rightarrow$ jede der Mengen $A_k$ ist beschränkt.\\
Mit dem Vollständigkeitsaxiom existiert für jedes $A_k$ ein Infinum. Sei etwa, $X_k = inf(A_k)$\\
$A_0 \supset A_1 \supset ... \supset A_{k-1} \supset A_k \supset ... \Rightarrow x_k \le x_{k+1} \forall k \ge 0$\\
d.h. Die Folge $(x_k)_{k \ge 0}$ ist monoton wachsen und durch $A$ nach oben beschränkt. Nach dem Monotoniekriterium ist die Folge $(x_k)_{k \ge 0}$ konvergent.
Sei etwa der $\lim\limits{k \toinf}(x_k) = z$. \\
Behauptung: $z$ ist Häufungspunkt von $(a_n)$.\\
I) Sei $\epsilon > 0$ , da $\lim{x_k} = z \Rightarrow \exists N \in \N$ mit $\abs{x_k - z} < \frac{\epsilon}{2} \forall n \ge N$
II) Da $x_k = inf(A_k) = inf(\{a_m : m \ge k\}) \Rightarrow \exists a_{k_m} \text{ mit } \abs{x_k - a_{k_m}} < \frac{\epsilon}{2}$.
%[GRAFIK: zahlengerade mit x_k, x_k + (\epsilon / 2) und element dazwischen:  a_k_m]
$\Rightarrow \abs{a_{k_m} - z} = \abs{a_{k_m} - x_k + x_k - z} \le  \abs{a_{k_m} - x_k} + \abs{x_k  -z} < \frac{\epsilon}{2} + \frac{\epsilon}{2} = \epsilon$\\
Also $\forall \epsilon > 0 \exists N \in \N \forall k \ge N\exists a_{k_m} \in (a_n)$
$\abs{a_{k_m} - z} < \epsilon$\\
d.h. die Teilfolge $(a_{k_m})_{m\ge 0}$ ist konvergent gegen $z$\\
Also $(a_{k_m})$ ist eine konvergente Teilfoge von der beschränkten Folge $(a_n)$.
\end{proof}
\end{satz}

\begin{bem}
Äquivalente Formulierungen:
Jede beschränkte Folge reeler Zahlen hat mindestens einen Häufungspunkt
Jede beschränkte Folge reeller Zahlen hat einen größten und einen kleinsten Häufungspunkt
\end{bem}

\begin{defi}[Cauchy-Folge] 
Eine Folge $(a_n)_{n \ge 0}$ heißt CAUCHY-Folge, wenn gilt:
$\forall \epsilon > 0 \exists N \in \N \forall n, m \ge N \abs{a_n - a_m} < \epsilon$
\end{defi}

\begin{satz}
Folgende Aussagen sind äquivalent
\begin{enumerate}
\item Die Folge $(a_n)$ ist konvergent
\item Die Folge $(a_n)$ ist eine Cauchy-Folge
\end{enumerate}
\begin{proof}
\"1) $\Rightarrow$ 2)\"\\
Sei $\lim{a_n} = a \Rightarrow \forall \epsilon > 0 \exists N : \forall m \ge N \abs{a_n a} < \frac{\epsilon}{2} \Rightarrow \forall n, m \ge N
\abs{a_n - a_m} = \abs{a_n - a + a - a_m} \le \abs{a_n - a }  + \abs{a_m - a} <  \frac{\epsilon}{2}+ \frac{\epsilon}{2} = \epsilon
\Rightarrow a_n$ ist eine Cauchy Folge
\"2) $\Rightarrow$ 1)\"\\
Jede Cauchy Folge ist beschränkt. Sei $\epsilon = 1 \Rightarrow \exists N \in N \forall n, m \ge N \abs{a_n - a_m} < 1 \Rightarrow \abs{a_n - a_N} < 1 \Rightarrow \abs{a_n} = \abs{a_n - a_N + a_N} \le \abs{a_n - a_N} + \abs{a_N} < 1 + \abs{a_N} \forall n  \ge N
\Rightarrow \forall n \in \N \abs{a_n} \le max\{\abs{a_0},...,\abs{a_{N-1}},\abs{a _N} + 1\} \Rightarrow (a_n)$ ist beschränkt.
Nach dem Satz von Bolzano-Weierstraß existiert eine konvergente Teilfolge $(a_{n_k})_{k \ge 0}$ sei $\lim{a_{n_k}} = a$.\\
Wir zeigen. $lim{a_n} = a.$\\
Sei $\epsilon > 0$. Wähle $m$ so groß, dass $\abs{a_m - a_n} < \frac{\epsilon}{2} \forall n,m \ge N und \abs{a_{n_k} - a} < \frac{\epsilon}{2} \forall k \ge N \Rightarrow \abs{a-a_n} = \frac{a - a_{n_k} + a_{n_k} - a_n} \le \abs{a- a_{n_k}} + \abs{a_{n_k} - a_n} < \frac{\epsilon}{2} + \frac{\epsilon}{2} = \epsilon$, da $n_k \ge n \ge N$
\end{proof}
\end{satz}

%% KORRIGIERE AB HIER!
\begin{bsp}[Verfahren zur Berechnung der Quadratwurzel]
Seien $a = 0$, $a_0 > 0$ reelle Zahlen. Wir definieren die Folge $(x_n)$ rekursiv.\\
\[x_0 = x_0\]
\[x_{n+1} =  \frac{1}{2}\left(x_n + \frac{a}{x_n}\right) \]
Wir zeigen: $(x_n)$ ist konvergent und $\lim{x_n} = x$ und $x^2 = a$.
\begin{proof}
\begin{enumerate}
\item $x_n > 0 \forall n \ge 0$
\IA{n=0} $x_0 > 0$ \\
\IV \(x_{n+1} =  \frac{1}{2}\left(x_n + \frac{a}{x_n}\right)\)
\IS{n \to n+1}
Sei $x_n > 0 \Rightarrow x_n+1 = 1/2 (x_n + a/x_n) > 0$ , da alle Teile $> 0$ sind.(mit under brace und overbrace
d.h. $(x_n)$ ist nach unten durch $0$ beschränkt.
\item $x_n^2 \ge a \forall n \ge 1$
denn $x_{n+1}^2 - a = \frac{1}{4}\left(x_n + \frac{a}{x_n}\right)^2  - a = ... a$ in klammer rein zieihen und ausrechnen $... = \frac{1}{4}$ (mals  $a$ quradrat) $\ge 0$
\item $(x_n)$ ist monoton fallend
$x_n - x_{n+1} = x_n - \frac{1}{2} \left(x_n + \frac{a}{x_n}\right) = (x_n$ in klammer rein zeihen und ausrehcnen $) = \frac{1}{(2x_n)(x_n^2 - a)} \ge 0 $ weil beides $\ge 0 (x_n > 0)
\Rightarrow x_>n >= x_{n+1}$
Nach dem Monotonie-Kriterium ist $(x_n)$ konvergent.
\item Sei $x= \liminf{n}{x_n} \Rightarrow x = \lim\limits_{n \toinf}{x_{n+1}} = \lim\limits_{n \toinf}{\frac{1}{2}\left(x_n + \frac{a}{x_n}\right)} = \frac{1}{2}\left(\lim\limits_{n \toinf}x_n + \frac{a}{\lim\limits_{n \toinf}x_n}\right) =  \frac{1}{2}\left(x + \frac{a}{x}\right) \Rightarrow 2x = x + a/x \Rightarrow \\
x = a/x \Rightarrow x^2 = a.$
\end{enumerate}
\end{proof}
\end{bsp}
Die positive Lösung der Gleichung $x^2 = a$ heißt die Quadratwurzeln von $a$. Wir Schreiben $x = \sqrt{a}$.

\subsection{Reihen}

\begin{defi}
Sei $(a_n)_{n\ge0}$ eine Folge reeller Zahlen. Sei weiters $S_N = \sum_{n = 0}^{N}{a_n}$ die $N$-te Partialsumme, dann heißt die Folge $(S_N)_{N \ge 0}$ der Partialsummen eine unendliche Reihe.\\
\[\text{Schreibweise}\suminf{n}{a_n}\]
Konvergiert die Folge \((S_N)\text{ mit }\liminf{N}{S_N} = s \text{, dann heißt } \suminf{n}{a_n} = s\)der Wert der Reihe.\\
Man sagt: Die Reihe \(\suminf{n}{a_n}\text{ konvergiert (}\suminf{n}{a_n} < \infty\)).
\end{defi}

\begin{bsp}
\begin{enumerate}
\item Die geometrische Reihe. Sei \(\abs{a} < 1 \Rightarrow \suminf{n}{a^n} = \frac{1}{1-a}\text{. Ist }\abs{a} \ge 1\text{, dann ist }\suminf{n}{a^n}\) divergent.
\begin{proof}
Die geometrische Summe: \(\sum_{n = 0}^{N}{a^n} = \frac{1-a^{N+1}}{1-a}\) dann:
\begin{math}
\IA{N=0} 1 = a^0 = (1-a) / (1- a) = 1\\
\IV \sum_{n = 0}^{N}{a^n} = \frac{1-a^{N+1}}{1-a}\\
\IS{N \to N+1}\\
\sum_{n = 0}^{N+1}{a^n} = a^{N+1} + \sum_{n = 0}^{N}{a^n} \overset{IV}{=} a^{N+1} + \frac{1-a^{N+1}}{1-a} = .... \text{selber} \\ %todo
\text{Sei }S_N = \sum_{n=0}^{N}{a^n} = \frac{1 - a^{N+1}}{1- a} \\
\text{Sei }\abs{a} < 1\text{. Dann folgt }\liminf{N}{a^N} = 0 \\
\Rightarrow \liminf{N}{S_N} = \liminf{n}{\frac{1-a^{N+1}}{1-a}} = \frac{1}{1-a}\\
\text{Sei }a \ge 1 \Rightarrow \sum_{n=0}^{N}{a^n} \ge \sum_{n=0}^{N}{1} = N+1 \longrightarrow \inf \\
\text{Sei }a \le -1 \Rightarrow a = -b \text{ mit }b \ge 1 \Rightarrow \sum_{n=0}^{N}{a^n} \ge \sum_{n=0}^{N}{(-1)^nb^n} \text{ divergent}
\end{math}
\end{proof}
\item Die harmonische Reihe: $\sum{n=1}{\infty}{1/n} = +\infty$
\begin{proof}
\begin{math}
S_{2^N} = \sum_{n = 1}^{2^N}{\frac{1}{n}} = 1 +\underbrace{\frac{1}{2}}_{=\frac{1}{2}} + \underbrace{\frac{1}{3} + \frac{1}{4}}_{=\frac{1}{2}} + \underbrace{\frac{1}{5} + \frac{1}{6} + \frac{1}{7} + \frac{1}{8}}_{=\frac{1}{2}} + \underbrace{\frac{1}{9} + ... + \frac{1}{16}}_{=\frac{1}{2}} + \underbrace{\frac{1}{17} + ..... + }_{=\frac{1}{2}} + \underbrace{\frac{1}{2^{N-1}+1}+ .... + \frac{1}{2^N}}_{=\frac{1}{2}} \ge 1 + n \frac{1}{2} > \frac{n}{2} \longrightarrow +\infty
\end{math}
Würde $(S_N)_{N\ge1}$ konvergieren, dann auch die Teilfolge $(S_{2^N})_{N\ge1}$, Da diese dievergiert, divergiert auch $(S_N)_N$
\end{proof}
\item $\Suminf{n}{\frac{1}{n(n+1)}} = 1$
\begin{proof}
\begin{math}
S_N = \sum_{n=1}^{N}{\frac{1}{n(n+1)}} = \sum_{n=1}^{N}{\frac{1}{n}} - \frac{1}{n+1} = \sum_{n=1}^{N}{\frac{1}{n}} - \sum_{n=1}^{N}{\frac{1}{n+1}} = 1+ \sum_{n=2}^{N}{\frac{1}{n}} - \sum_{n=2}^{N+1}{\frac{1}{n}} = 1+ \sum_{n=2}^{N}{\frac{1}{n}} - \sum_{n=2}^{N}{\frac{1}{n}} + \frac{1}{N+1} = 1 + \frac{1}{N+1} \longrightarrow 1
\end{math}
\end{proof}
\end{enumerate}
\end{bsp}

\begin{satz}
Seien \suminf{n}{a_n} und \suminf{n}{b_n} zwei konvergente Reihen und \(\lambda \in \R\)
Dann ist auch \(\suminf{n}{\lambda a_n + b_n}\) konvergent und \( \suminf{n}{\lambda a_n + b_n} = \lambda \suminf{n}{a_n} + \suminf{n}{b_n}\)
\begin{proof} folgt auf grund der Rechenregeln für konvergente Folgen.\end{proof}
\end{satz}

\begin{satz}[Cauchy-Kriterium für Reihen]
Die Reihe \(\suminf{k}{a_k}\) ist konvergent, genau dann wenn gilt:
\[\forall \epsilon > 0 \sp \exists N(\epsilon) \in \N \quad \forall n \ge m \ge N \qquad \abs{\sum{k=m}{n}{a_k}} < \epsilon \qquad (\star)\]
\begin{proof}
\(S_n - Sm = \sum{k=0}{n}{a_k} - \sum{k=0}{m}{a_k} = \sum{k=m}{n}{a_k}\). ($\star$) bedeutet die $(S_n)_n$ ist eine Cauchy-Folge $\Leftrightarrow (S_n)_n$ ist konvergent
\end{proof}
\end{satz}


\begin{korr}
Ist $\suminf{k}{a_k}$ konvergent $\Rightarrow \lim\limits_{k \toinf}{a_k} = 0$.
\begin{proof}
\begin{math}
a_n = \sum{k = m}{n}{a_k}\text{. Da } \suminf{k}a_k < \infty \overset{\Rightarrow}_{Cauchy-kriterium} \forall \epsilon > 0 \sp \exists N \in \N \quad \forall n \ge N \abs{a_N} = \abs{ \sum{k = m}{n}{a_k}} < \epsilon  \Rightarrow \lim\limits_{n \toinf}{a_n} = 0
\end{math}
\end{proof}
\end{korr}

\begin{bem}
Die Umkehrung des Korrolars gilt nicht. z.B. \(\liminf{n}{1/n} = 0\text{ aber }\suminf{1/n} = \infty \)harmonische Reihe.
\end{bem}

\begin{defi}
Die Reihe \suminf{k}{a_k} heißt absolut konvergent, wenn wenn die Reihe \suminf{k}{\abs{a_k}} konvergiert.
\end{defi}

\begin{satz}
Jede absolut konvergente Reihe ist auch konvergent.
\begin{proof}
Sei \suminf{k}{\abs{a_k}} < \infty  \overset{\Rightarrow}_{Cauchy-kriterium} \forall \epsilon > 0 \sp \exists N \in \N \quad \forall n \ge m \ge N \quad \abs{\sum{k=m}{n}{\abs{a_k}}} < \epsilon \overset{\Rightarrow}_{Dreiecksungleichung} \abs{\sum{k=m}{n}{a_k}} \le \abs{\sum{k=m}{n}{\abs{a_k}}} < \epsilon \forall n \ge m \ge N
\overset{\Rightarrow}_{Cauchy-Kriterium} \suminf{k}{a_k} ist konvergent.
\end{proof}
\end{satz}

\begin{bem}
Die Umkehrung des Satzes gilt nicht. zB kann man zeigen, dass die Reihe \suminf{k}{(-1)^k(1/k)} konvergiert. aber die Reihe \suminf{k}{\abs{(-1)^k(1/k)}} = \suminf{k}{(1/k)} = \infty
\end{bem}

\begin{satz}[Majoranten-Kriterium]
Sei \suminf{k}{b_k} konvergent mit b_k \ge 0 \forall k \ge N_0.
Sei (a_k)_{k=0}^{\infty} eine Folge mit \abs{a_k} \le b_k  \forall k \ge N_0
\Rightarrow \suminf{k}{a_k} ist absolut konvergent.
\begin{proof}
Sei \suminf{k}{b_k} < \infty und b_k > 0 \overset{\Rightarrow}_{Cauchy-Kriterium} \forall \epsilon > 0 \sp \exists N \in \N \quad \forall n \ge m \ge N \quad \abs{\sum{k=m}{n}{b_k}} < \epsilon \overset{\Rightarrow}_{\abs{a_k} \le b_k} \abs{\sum{k=m}{n}{\abs{a_k}}} \le \abs{\sum{k=m}{n}{b_k}} < \epsilon \forall n \ge m \ge N
\overset{\Rightarrow}_{Cauchy-Kriterium} \suminf{k}{\abs{a_k}} ist konvergent. \Rightarrow \suminf{k}{a_k} ist absolut konvergent.
\end{proof}
\end{satz}

\begin{korr}[Minoranten-Kriterium]
Sei \suminf{k}{b_k} divergent mit b_k \ge 0 \forall k \ge N_0.
und (a_k)_{k=0}^{\infty} eine Folge mit \abs{a_k} \ge b_k  \forall k \ge N_0
\Rightarrow \suminf{k}{a_k} ist auch divergent.
\begin{proof}
Wäre \suminf{k}{a_k} konvergent, dann wäre nach dem Majoranten-Kriterium \suminf{k}{b_k} konvergent, da \(\abs{b_k} \le a_k\). \WSP
\end{proof}
\end{korr}

\begin{satz}[Quotienten-Kriterium]
Sei \suminf{n}{a_n} eine Reihe  mit a_n \ne 0 \forall n \ge n_0
Existiert eine reelle Zahl q mit 0 < q < 1 sodass \abs{\frac{a_{n+1}}{a_n}} \le q < 1 \forall n \ge n_0
\Rightarrow \suminf{n}{a_n} ist absolut konvergent.
\begin{proof}
Sei \abs{\frac{a_{n+1}}{a_n}} \le q < 1  \forall n \ge 0 (o.B.d.A.) \Rightarrow \abs{a_n+1} \le q \abs{a_n} \Rightarrow \abs{a_n} \le q\abs{a_{n-1}} \le q^2 \abs{a_{n-2}} \le ... \le q^n \abs{a_0}.
Also \abs{a_n} \le q^n \abs{a_0}, da \suminf{n}{\abs{a_n}} \le \suminf{n}{q^n \abs{a_0}} = \abs{a_0} \suminf{n}{q^n} = \abs{a_0} \frac{1}{1-q}, denn 0 < q < 1, geometrische Reihe.
\Rightarrow aus dem Majoranten-Kriterium folgt \suminf{n}{a_n} ist absolut konvergent.
\end{proof}
\end{satz}

\begin{korr}[einfaches Quotienten-Kriterium]
Sei a_n \ne 0 \fa n > n_0 und existiert \lim\limits_{n \toinf}\abs{\frac{a_n+1}{a_n}} und ist \lim\limits_{n \toinf}\abs{\frac{a_n+1}{a_n}} < 1
\Rightarrow \suminf{n}{a_n} ist absolut konvergent.
\begin{proof}
Sei\lim\limits_{n \toinf}\abs{\frac{a_n+1}{a_n}} = \alpha < 1 \\
Sei \espilon = \frac{1 - \alpha}{2} > 0 \Rightarrow \exists N \sp \forall n \ge N \quad \abs{\abs{a_{n+1}}{a_n}} - \alpha} < \epsilon  = \frac{1-\alpha}{2}
\Rightarrow \abs{a_(n+1) / a_n} < (1-alpha)/2) + alpha = (1+alpha)/2 \overset{<}{da \alpha < 1} 1+1/2 = 1
Sei q = (1 + \alpha)/2 < 1 und |a_(n+1)/a_n| < q < 1
Nach dem Quotienten-Kriterium ist \suminf{a_n} absolut konvergent
\end{proof}
\end{korr}

\begin{bsp}
\begin{enumerate}
\item \suminf{n}{\frac{1}{n^k} < \infty \qqad \forall k \ge 2 \\
[Bemerkung: Die Konvergenz gilt auch \forall k \in \R, k > 1 ohne Beweis]
\begin{proof}
1/n^k \le 1/n^2 \qqad \forall k \ge 2
und 1\n^2 \le 2/(n(n+1)), denn \Leftrightarrow 2n^2 \ge n(n+1) \Leftrightarrow n^2 \ge n \Leftrightarrow n \ge 1\\
\Rightarrow 1/n^k \le 2/(n(n+1)) \forall k \ge 2
und \suminf{n}{2/(n(n+1))} = 2 \suminf{n}{1/(n(n+1))} = 2 * 1 = 2
Aus dem majoranten-Kriterium folgt \suminf{n}{1/n^k} < \infty \froall k \ge 2\\
Frage: Wie sind die Werte der Reihe \suminf{n}{1/n^k} = ? \froall k \ge 2
Euler: \suminf{n}{1/n^2}  = (\pi^2)/6, \suminf{n}{1/n^4}  = (\pi^4)/90,, .... , \suminf{n}{1/n^{2k}} = C_k \pi^{2k}
Aber: \suminf{n}{1/n^3} \in \R \setminus \Q, \suminf{n}{1/n^5} = ?,... \suminf{n}{1/n^{2k+1}} = ?   
\end{proof}
\item Die Reihe \suminf{n}{n^2/2^n} ist konvergent.
\begin{proof}[Quotienten-Kriterium]
\abs{a_{n+1}/ a_n} = \frac{\frac{(n+1)^2}{2^{n+1}}{\frac{n^2}{2^n}} = \frac{2^n(n+1)^2}{2^{n+1}n^2} = 1/2 * ((n+1)/n)^2 = 1/2 (1/(1+1/n))^2 \longrightarrow 1/2 < 1
\end{proof}
\item Die Exponentialfunktion Die Reihe \suminf{k}\frac{x^k}{k!} ist für jedes x \in \R absolut konvergent
\begin{proof}[Quotienten-Kriterium]
\abs{a_{k+1}/a_k} = \abs{\frac{x^{k+1}}{(k+1)!}/\frac{x^k}{k!}} = \frac{\abs{x^{k+1}} * k!}{\abs{x^k} (k+1)!} = \frac{\abs{x}}{k+1} \longrightarrow mit k-> inf 0 \forall x \in \R
\Rightarrow \ifnsum{k}\frac{x^k}{k!} ist absolut konvergent.
\end{proof}
\end{enumerate}
\end{bsp}

\begin{bem}
1. Für k= 1 ist die harmonische Reihe \suminf{n]{1/n} divergent.
2. Das Quotienten-Kriterium ist hier nicht anwendbar, denn \\
\sum{1/n} \qqad a_{n+1}/a_n = 1/(1 + 1/n) \longrightarrow 1 \not < 1
\sum{1/n^2} \qqad a_{n+1}/a_n = 1/((1 + 1/n)^2) \longrightarrow 1 \not < 1
\end{bem}

\begin{defi}
Die Funktion exp: \R \to \R mit exp(x) \mapsto \e^x = \suminf{n}\frac{x^n}{n!} heißt Exponentialfunktion
Die Zahl \e = exp(0) = \suminf{n}\frac{1^n}{n!} heißt Euler'sche Zahl.
\end{defi}

\begin{bem}
Wir werden später zeigen: \e = \frac{1^n}{n!} = \lim\limits{n \toinf}{\left(1 + \frac{1}{n}\right)^n\\
\e \approx 2,71828...
\end{bem}



























%soll am ende stehen7
\newpage
\section{Test}
zum Formeln raus kopieren
\begin{math}
\lim\limits_{n \toinf}{a_n}\\
\sum_{n = 0}^{\infty}{a_n}
\end{math}

$(a_n)$


\end{document}
